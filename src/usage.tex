\section{Usage}\label{sec:usage}

It is possible to use this template locally or optionally get started right away online via Google Project
IDX\parencite{noauthor_project_nodate}.

When using GitHub to host your repository, any changes you push will automatically produce and release the matching PDF
document.

Configuration of LaTeX is done in through the Nix flake file.
Check it out to add TeX Live packages or any other dependency you may require.

\subsection{Local usage}\label{subsec:local-usage}

To use this template locally, make sure to have Nix\parencite{noauthor_download_nodate} installed and ready to go.
You probably do not want to have to use Nix as the root user, so make sure to ensure the multi-user mode is set up
correctly\parencite{noauthor_multi-user_nodate}.
After completing the Nix installation, the LaTeX project compilation is as simple as running

\begin{verbatim}
nix --experimental-features 'nix-command flakes' run
\end{verbatim}

which can be shortened to

\begin{verbatim}
nix run
\end{verbatim}

when enabling Flakes by default\parencite{noauthor_flakes_nodate}.
The result matches the result of the continues delivery pipeline.

Integration with your text editor of choice can normally be achieved by pointing to the helper scripts in the binary
folder of this template.
This has been tested for JetBrains IDEs using the TeXiFy-IDEA plugin\parencite{noauthor_hannah-stentexify-idea_2025}.

If you cannot install Nix on to your machine, you can use the development
container\parencite{noauthor_development_nodate} configuration instead.
The configuration also includes a minimal configuration for a VSCode and IntelliJ setup.

\subsection{Online usage}\label{subsec:online-usage}

To get started using any of the supported online code environment providers is just a matter of importing this template.
You might be able to use any remote development environment provider that supports development containers.
